\usepackage{multirow}
\usepackage{titlesec}
\usepackage{fancyhdr}
\usepackage{graphics}
\usepackage{siunitx}
\sisetup{per-mode = symbol% Use V/cm instead of V cm⁻¹
,exponent-product=\cdot% Write the exponents like 5·10³, not 5x10³
,range-phrase= \ldots % Write the exponents like 5·10³, not 5x10³
}
\DeclareSIUnit[number-unit-product = {}]\var{var} % Volt-Ampère reactive
\usepackage{enumitem} % Enumitem for easier customization of enumerate lists
\usepackage{../templates/ET_Ueb_Header}
\usepackage{epsfig,subfigure,cite}
%
\setlength{\textwidth}{15cm}%
\setlength{\oddsidemargin}{0.3cm}
%
\renewcommand{\textfraction}{0.1}
\titleformat{\section}[hang]{\normalfont\Large\bfseries}{Aufgabe
\thesection}{.8em}{}
\titleformat{\subsection}[hang]{\normalfont\large\bfseries}{Teil
\thesubsection}{.8em}{}
\renewcommand\thesubsection{\arabic{section}\Alph{subsection}}
\setlist[enumerate,1]{label=\thesubsection.\arabic*),ref=\thesubsection.\arabic*)}
%
\let\cpx\underline %We use underlines as notation for complex variables
\let\peak\hat %The amplitude (peak) of a alternating value (e.g. û)
\newcommand\pcpx[1]{\peak{\cpx{#1}}} % Peak of complex. Since this is used quite often as "Komplexer Zeiger", we define a macro for it
\let\avg\overline %The time-average of a value
\newcommand{\abs}[1]{\left\lvert#1\right\rvert} %The absolute value of a signal, i.e. \abs{u} → |u|
\newcommand{\dd}{\;\mathrm{d}} % The Volume element of an integral, \int ... \dd
\renewcommand{\d}{\operatorname{d}\!} % The differential d for \frac{\d}{\d t}
\renewcommand{\j}{\operatorname{j}\!} % The imaginary unit
\newcommand{\conj}[1]{{#1}^\ast} % Complex conjugation
\newcommand{\mat}[1]{\mathbf{#1}} % A matrix
\newcommand{\cmat}[1]{\mat{\cpx{#1}}} % A complex matrix

\documentclass[a4paper,12pt]{article}
\usepackage[utf8]{inputenc}
\usepackage{amsmath}
\usepackage{geometry}
\geometry{margin=2.5cm}

\title{Kernkompetenz – Bodeplots zeichnen}
\date{}
\author{Rares Sahleanu}

\begin{document}

\maketitle

\noindent
Zeichnen Sie jeweils den Bode-Plot der folgenden Übertragungsfunktionen \( H(s) \).\\[1em]

\newcommand{\aufgabe}[2]{
  \parbox[t]{0.48\textwidth}{
    \textbf{(#1)} \quad \Large\( #2 \)
  }
}

% --- Aufgaben (a) bis (p): Einfach bis mittel ---

\aufgabe{a}{H(s) = \frac{1}{s + 1}}
\hfill
\aufgabe{b}{H(s) = \frac{10}{s + 10}}

\vspace{1.5em}

\aufgabe{c}{H(s) = \frac{s + 1}{s + 10}}
\hfill
\aufgabe{d\textbf{$\star$}}{H(s) = \frac{10(1-s)}{s + 10}}

\vspace{1.5em}

\aufgabe{e\textbf{$\star$}}{H(s) = \frac{(-1+1j)}{\sqrt{2}(s + 1)^2}}
\hfill
\aufgabe{f}{H(s) = \frac{-1000}{(s + 1)(s + 100)}}

\vspace{1.5em}

\aufgabe{g}{H(s) = \frac{\sqrt{2}100s}{s + 1}}
\hfill
\aufgabe{h}{H(s) = \frac{\sqrt{2} 10s^2}{s - 1}}

\vspace{1.5em}

\aufgabe{i}{H(s) = \frac{(s + 1)}{(s + 10)^2}}
\hfill
\aufgabe{j}{H(s) = \frac{s + 1}{s^2 + 2s + 1}}

\vspace{1.5em}

\aufgabe{k}{H(s) = \frac{100(s + 1)}{s^2 + 20s + 100}}
\hfill
\aufgabe{l}{H(s) = \frac{s^2 - 100}{(s + 1)}}

\vspace{1.5em}

\aufgabe{m}{H(s) = \frac{\sqrt{202}10s}{(s + 1)(s + 10)}}
\hfill
\aufgabe{n\textbf{$\star$}}{H(s) = \frac{s(0.1-s)(s + 10)}{(s + 1)^2}}

\vspace{1.5em}

\aufgabe{o}{H(s) = \frac{1}{s}}
\hfill
\aufgabe{p\textbf{$\star$}}{H(s) = \frac{100}{(s^2 + 1s + 100)}}


\vspace{1.5em}

% --- Aufgaben (q) bis (t): schwerer, nicht faktorisiert ---

\aufgabe{q\textbf{$\star$}\textbf{$\star$}}{H(s) = \frac{s^2 + 4}{s(s^2 + 10s + 100)}}
\hfill
\aufgabe{r}{H(s) = \frac{s^2 + 2s + 10}{s^2 + 2s + 10}}

\vspace{1.5em}

\aufgabe{s}{H(s) = \frac{4}{s^2 - 4}}
\hfill
\aufgabe{t\textbf{$\star$}}{H(s) = \frac{-1000(s + 2)^2}{4(s + 1)^3(s + 10)}}

\hfill

\aufgabe{u}{H(s) = \frac{2s}{s^2 + 2s + 1}}

\end{document}
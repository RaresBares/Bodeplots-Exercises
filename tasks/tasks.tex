\documentclass[a4paper,12pt]{article}

% --- Präambel (wie main, inkl. Header-Stil) ---
\usepackage{graphicx}
\usepackage{amsmath}
\usepackage{siunitx}
\sisetup{per-mode=symbol,exponent-product=\cdot,range-phrase=\ldots}
\usepackage{tikz,pgfplots}
\pgfplotsset{compat=1.18}
\usepackage{titlesec}
\usepackage{enumitem}

% Header-/Titelseitenstil mit Logos; Pfad von ./tasks aus:
\usepackage{../tex/ET_Ueb_Header_2Ebenen} % falls die Datei ET_Ueb_Header.sty heißt: ../tex/ET_Ueb_Header
% zusätzliche Grafikpfade (Root & tasks):
\graphicspath{{../}{../templates/}{../fig/}{./}{./templates/}{./fig/}}

% Formatierung (wie main)
\renewcommand\thesection{\Alph{section}}
\titleformat{\section}[hang]{\normalfont\Large\bfseries}{Aufgabe \thesection)}{.8em}{}
\renewcommand\thesubsection{\thesection.\arabic{subsection}}
\titleformat{\subsection}[hang]{\normalfont\large\bfseries}{\thesubsection}{.8em}{}
\setlist[enumerate,1]{label=\thesubsection),ref=\thesubsection)}

% Makros (wie main)
\let\cpx\underline
\let\peak\hat
\newcommand\pcpx[1]{\peak{\cpx{#1}}}
\let\avg\overline
\newcommand{\abs}[1]{\left\lvert#1\right\rvert}
\newcommand{\dd}{\;\mathrm{d}}
\renewcommand{\d}{\operatorname{d}\!}
\renewcommand{\j}{\operatorname{j}\!}
\newcommand{\conj}[1]{{#1}^\ast}
\newcommand{\mat}[1]{\mathbf{#1}}
\newcommand{\cmat}[1]{\mat{\cpx{#1}}}
\makeatletter
\@ifundefined{Header}{\let\Header\makeheader}{}
\makeatother

\title{Kernkompetenz – Bodeplots zeichnen}
\author{}
\date{}

\begin{document}

% Titelseite 1:1 wie in main; Logos kommen aus dem Header-Stil via \thispagestyle{firstpage}
\begin{titlepage}
  \thispagestyle{firstpage}
  \vspace*{4cm}
  \begin{center}
    {\Large\bfseries Netzwerke und Schaltungen II, D-ITET\par}
    \vspace{3mm}
    {\Huge\bfseries Bodeplots — Aufgaben\par}
    \vspace{9mm}
  \end{center}
  \vfill
\end{titlepage}

% kein \maketitle

\noindent
Zeichnen Sie jeweils den Bodeplots für Phasen- und Magnitudenverläufe der folgenden Übertragungsfunktionen \( H(s) \).\\[1em]

\newcommand{\aufgabe}[2]{%
  \parbox[t]{0.48\textwidth}{\textbf{(#1)} \quad \Large\( #2 \)}%
}

% --- Aufgabenliste (dein Inhalt unverändert) ---

\aufgabe{a}{H(s) = \frac{1}{s + 1}}
\hfill
\aufgabe{b}{H(s) = \frac{10}{s + 10}}

\vspace{1.5em}

\aufgabe{c}{H(s) = \frac{s + 1}{s + 10}}
\hfill
\aufgabe{d\textbf{$\star$}}{H(s) = \frac{10(1-s)}{s + 10}}

\vspace{1.5em}

\aufgabe{e\textbf{$\star$}}{H(s) = \frac{(-1+1j)}{\sqrt{2}(s + 1)^2}}
\hfill
\aufgabe{f}{H(s) = \frac{-1000}{(s + 1)(s + 100)}}

\vspace{1.5em}

\aufgabe{g}{H(s) = \frac{100s}{s + 1}}
\hfill
\aufgabe{h}{H(s) = \frac{10\sqrt{2}\, s^2}{s - 1}}

\vspace{1.5em}

\aufgabe{i}{H(s) = \frac{s + 1}{(s + 10)^2}}
\hfill
\aufgabe{j}{H(s) = \frac{s + 1}{s^2 + 2s + 1}}

\vspace{1.5em}

\aufgabe{k}{H(s) = \frac{100(s + 1)}{s^2 + 20s + 100}}
\hfill
\aufgabe{l}{H(s) = \frac{s^2 - 100}{s + 1}}

\vspace{1.5em}

\aufgabe{m}{H(s) = \frac{10\sqrt{202}\, s}{(s + 1)(s + 10)}}
\hfill
\aufgabe{n\textbf{$\star$}}{H(s) = \frac{s(0.1-s)(s + 10)}{(s + 1)^2}}

\vspace{1.5em}

\aufgabe{o}{H(s) = \frac{1}{s}}
\hfill
\aufgabe{p\textbf{$\star$}}{H(s) = \frac{100}{s^2 + s + 100}}

\vspace{1.5em}

\aufgabe{q\textbf{$\star$}\textbf{$\star$}}{H(s) = \frac{s^2 + 4}{s(s^2 + 10s + 100)}}
\hfill
\aufgabe{r}{H(s) = \frac{s^2 + 2s + 10}{s^2 + 2s + 10}}

\vspace{1.5em}

\aufgabe{s}{H(s) = \frac{4}{s^2 - 4}}
\hfill
\aufgabe{t\textbf{$\star$}}{H(s) = \frac{-1000(s + 2)^2}{4(s + 1)^3(s + 10)}}

\vspace{1.5em}

\aufgabe{u}{H(s) = \frac{2s}{s^2 + 2s + 1}}

\end{document}
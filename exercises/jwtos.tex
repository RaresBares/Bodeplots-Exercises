\section*{Einleitung: \(s\) und \(j\omega\)}

Für sinusförmige stationäre Signale fällt die Laplace- mit der Fourier-Transfor- mierten auf der imaginären Achse zusammen; eine Abklinghülle ist nicht nötig, daher setzt man \(\sigma=0\) und damit
\[ s = j\omega. \]
Der Frequenzgang wird zwar als \(H(j\omega)\) ausgewertet, aber die Schreibweise in \(s\) ist kompakter: Standardformen wie \(1+sT\), \(1/(1+sT)\), \(sL\), \(1/(sC)\) sind sofort lesbar und direkt faktorisierbar. Vorgehen: in \(s\) modellieren und faktorisieren, um Magnituden/Phasen explitizit auszurechnen, am Ende \(s\to j\omega\) einsetzen.


\subsection*{Beispiele}

$$ H(s)=\frac{1}{1+sT}\qquad\Rightarrow\qquad H(j\omega)=\frac{1}{1+j\omega T} $$

$$ H(s)=\frac{sT}{1+sT}\qquad\Rightarrow\qquad H(j\omega)=\frac{j\omega T}{1+j\omega T} $$

$$ H(s)=\frac{1}{sT}\qquad\Rightarrow\qquad H(j\omega)=\frac{1}{j\omega T} $$

$$ H(s)=sT\qquad\Rightarrow\qquad H(j\omega)=j\omega T $$

$$ H(s)=\frac{\omega_0^2}{s^2+2\zeta \omega_0 s+\omega_0^2}\qquad\Rightarrow\qquad H(j\omega)=\frac{\omega_0^2}{-\omega^2+j\,2\zeta \omega_0 \omega+\omega_0^2} $$

$$ H(s)=\frac{1+sT_{z}}{1+sT_{p}}\qquad\Rightarrow\qquad H(j\omega)=\frac{1+j\omega T_{z}}{1+j\omega T_{p}} $$

\newpage